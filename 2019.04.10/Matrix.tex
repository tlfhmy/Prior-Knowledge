\documentclass[a4paper,12pt]{ctexart}
\usepackage{amsmath,amssymb,amsthm}

\begin{document}
	\subsection{矩阵的定义}
		\subparagraph{矩阵的来源}
		矩阵起源于线性方程组的求解,像简单的二元一次方程组
		$
		\left\{
		\begin{array}{rl}
		a_{11}x + a_{12}y &= b_1\\
		a_{21}x + a_{22}y &= b_2
		\end{array}
		\right.
		$
		我们在求解的过程仅需要对其系数进行操作,于是乎,变量符号$x,y$对于我们来说,在计
		算过程中,并不显得十分必要;因此,我们可以略掉其变量名而将系数写在一对方括号之
		中,以便于进行计算。从该二元方程组我们可以得到如下形式的系数“方块”
		$$
		\left[
		\begin{array}{rl}
		a_{11} & a_{12}\\
		a_{21} & a_{22}
		\end{array}
		\right]
		=
		\left[
		\begin{array}{rl}
		b_1\\
		b_2
		\end{array}
		\right]
		$$
		类似的,我们可以从$n$元方程组(不一定有$n$个方程式)
		$$
		\left\{
		\begin{array}{rl}
		&a_{11}x_1 + a_{12}x_2 + \cdots + a_{1n}x_n = b_1\\
		&a_{21}x_1 + a_{22}x_2 + \cdots + a_{2n}x_n = b_2\\
		&\vdots \\
		&a_{m1}x_1 + a_{m2}x_2 + \cdots + a_{mn}x_n =b_m
		\end{array}
		\right.
		$$
		得到系数“方块”
		$$
		\left[
		\begin{array}{cccc}
		a_{11} & a_{12} & \cdots & a_{1n}\\
		a_{21} & a_{22} & \cdots & a_{2n}\\
		\vdots & \vdots & \cdots & \vdots\\
		a_{m1} & a_{m2} & \cdots & a_{mn}
		\end{array}
		\right]
		=
		\left[
		\begin{array}{rl}
		b_1\\
		b_2\\
		\vdots \\
		b_m
		\end{array}
		\right]
		$$
		于是,为求得方程组的解集,一般来说我们需要进行两种操作:一,将某一行乘以一个非零
		常数并不改变方程组的解集;二,将某一行的非零常数倍加到另一行,也不改变方程组的解
		集。这样,对应到我们新得到的“方块中”,就是将某一行乘以非零常数倍,或者将某一行的
		常数倍加到另一行之后,得到的新“方块”在方程组有同样的解集的意义下等价。至此,我们
		已经抽象出了矩阵的基本概念,接下来我们正式定义矩阵。
		
		\subparagraph{矩阵的定义}		
		在数学上,一个$m \times n$矩阵,是一个由$m$行,每行$n$个数字的列表组成。在矩
		阵上我们定义定义加法和乘法:
		\begin{enumerate}
		\item{加法:}\\
		并非任意种类的矩阵都可以进行加法运算,必须是同种类型的矩阵才可以,即,假设我们用记
		号$M_{m \times n}$表示所有的$m \times n$矩阵,那么对任意的矩阵$A,B$,当且仅
		当存在某个$M_{m \times n}$包含他们二者时,$A+B$才有意义;用数学符号表示为:
		$\exists M_{m \times n},A \in M_{m \times n} \& B \in M_{m \times n} 
		\Leftrightarrow \exists A+B$.直观的看来就是,两个矩阵相同位置的元素相加,所
		得到的新矩阵就是矩阵的和。
		\[		
		\left[
		\begin{array}{cccc}
		a_{11} & a_{12} & \cdots a_{1n}\\
		a_{21} & a_{22} & \cdots a_{2n}\\
		\vdots \\
		a_{m1} & a_{m2} & \cdots a_{mn}
				
		\end{array}		
		\right]		
		+
		\left[
		\begin{array}{cccc}
		b_{11} & b_{12} & \cdots b_{1n}\\
		b_{21} & b_{22} & \cdots b_{2n}\\
		\vdots \\
		b_{m1} & b_{m2} & \cdots b_{mn}
				
		\end{array}		
		\right]
		\]
		\[
		=
		\left[
		\begin{array}{cccc}
		a_{11}+b_{11} & a_{12}+b_{12} & \cdots a_{1n}+b_{1n}\\
		a_{21}+b_{21} & a_{22}+b_{22} & \cdots a_{2n}+b_{2n}\\
		\vdots \\
		a_{m1}+b_{m1} & a_{m2}+b_{m2} & \cdots a_{mn}+b_{mn}
				
		\end{array}		
		\right]						
		\]
		\item{乘法:}\\
		
		\end{enumerate}
		
\end{document}