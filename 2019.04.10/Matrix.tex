\documentclass[a4paper,12pt]{ctexart}
\usepackage{amsmath,amssymb,amsthm}
\usepackage{indentfirst}
\setlength{\parindent}{2em} 
\begin{document}
	\subsection{矩阵的定义}
		\subparagraph{矩阵的来源}
		矩阵起源于线性方程组的求解,像简单的二元一次方程组
		$
		\left\{
		\begin{array}{rl}
		a_{11}x + a_{12}y &= b_1\\
		a_{21}x + a_{22}y &= b_2
		\end{array}
		\right.
		$
		我们在求解的过程仅需要对其系数进行操作,于是乎,变量符号$x,y$对于我们来说,在计
		算过程中,并不显得十分必要;因此,我们可以略掉其变量名而将系数写在一对方括号之
		中,以便于进行计算。从该二元方程组我们可以得到如下形式的系数“方块”
		$$
		\left[
		\begin{array}{rl}
		a_{11} & a_{12}\\
		a_{21} & a_{22}
		\end{array}
		\right]
		=
		\left[
		\begin{array}{rl}
		b_1\\
		b_2
		\end{array}
		\right]
		$$
		类似的,我们可以从$n$元方程组(不一定有$n$个方程式)
		$$
		\left\{
		\begin{array}{rl}
		&a_{11}x_1 + a_{12}x_2 + \cdots + a_{1n}x_n = b_1\\
		&a_{21}x_1 + a_{22}x_2 + \cdots + a_{2n}x_n = b_2\\
		&\vdots \\
		&a_{m1}x_1 + a_{m2}x_2 + \cdots + a_{mn}x_n =b_m
		\end{array}
		\right.
		$$
		得到系数“方块”
		$$
		\left[
		\begin{array}{cccc}
		a_{11} & a_{12} & \cdots & a_{1n}\\
		a_{21} & a_{22} & \cdots & a_{2n}\\
		\vdots & \vdots & \cdots & \vdots\\
		a_{m1} & a_{m2} & \cdots & a_{mn}
		\end{array}
		\right]
		=
		\left[
		\begin{array}{rl}
		b_1\\
		b_2\\
		\vdots \\
		b_m
		\end{array}
		\right]
		$$
		\indent
		于是,为求得方程组的解集,一般来说我们需要进行两种操作:一,将某一行乘以一个非零
		常数并不改变方程组的解集;二,将某一行的非零常数倍加到另一行,也不改变方程组的解
		集。这样,对应到我们新得到的“方块中”,就是将某一行乘以非零常数倍,或者将某一行的
		常数倍加到另一行之后,得到的新“方块”在方程组有同样的解集的意义下等价。\\
		\indent
		至此,我们已经抽象出了矩阵的基本概念,接下来我们正式定义矩阵。
		
		\subparagraph{矩阵及其运算的定义}		
		在数学上,一个$m \times n$矩阵,是一个由$m$行,每行$n$个数字的列表组成。行列
		数相等,即$m=n$的矩阵被称为方阵,方阵对应很多独特的性质,今后将会谈到。在矩
		阵上我们定义定义加法和乘法:
		\begin{enumerate}
		\item{加法:}\\
		\indent 并非任意种类的矩阵都可以进行加法运算,必须是同种类型的矩阵才可以,即,
		假设我们用记号$M_{m \times n}$表示所有的$m \times n$矩阵,那么对任意的矩阵$A,B
		$,当且仅当存在某个$M_{m \times n}$包含他们二者时,$A+B$才有意义;用数学符号表示
		为:$\exists M_{m \times n},A \in M_{m \times n} \& B \in M_{m \times 
		n} \Leftrightarrow \exists A+B$.直观的看来就是,两个矩阵相同位置的元素相加,所
		得到的新矩阵就是矩阵的和。
		\[		
		\left[
		\begin{array}{cccc}
		a_{11} & a_{12} & \cdots a_{1n}\\
		a_{21} & a_{22} & \cdots a_{2n}\\
		\vdots \\
		a_{m1} & a_{m2} & \cdots a_{mn}
				
		\end{array}		
		\right]		
		+
		\left[
		\begin{array}{cccc}
		b_{11} & b_{12} & \cdots b_{1n}\\
		b_{21} & b_{22} & \cdots b_{2n}\\
		\vdots \\
		b_{m1} & b_{m2} & \cdots b_{mn}
				
		\end{array}		
		\right]
		\]
		\[
		=
		\left[
		\begin{array}{cccc}
		a_{11}+b_{11} & a_{12}+b_{12} & \cdots a_{1n}+b_{1n}\\
		a_{21}+b_{21} & a_{22}+b_{22} & \cdots a_{2n}+b_{2n}\\
		\vdots \\
		a_{m1}+b_{m1} & a_{m2}+b_{m2} & \cdots a_{mn}+b_{mn}
				
		\end{array}		
		\right]						
		\]
		\indent
		很明显,能够进行加法的矩阵$A$和$B$,$A+B=B+A$,因此矩阵的加法满足交换律。
		\item{乘法:}\\
		类似于加法,矩阵的乘法对矩阵的类型也有要求。我们假定$A\in M_{m \times n}$及
		$B \in M_{l \times k}$,当且仅当$n=l$时,$A \cdot B$才有意义,并且得到
		一个$n$行$k$列的新矩阵$C \in M_{m \times k}$。后面我会详细讨论这样定义矩
		阵乘法的实际用途。
		\[		
		\left[
		\begin{array}{cccc}
		a_{11} & a_{12} & \cdots a_{1n}\\
		a_{21} & a_{22} & \cdots a_{2n}\\
		\vdots \\
		a_{m1} & a_{m2} & \cdots a_{mn}
				
		\end{array}		
		\right]		
		\cdot
		\left[
		\begin{array}{cccc}
		b_{11} & b_{12} & \cdots b_{1k}\\
		b_{21} & b_{22} & \cdots b_{2k}\\
		\vdots \\
		b_{n1} & b_{n2} & \cdots b_{nk}
				
		\end{array}		
		\right]				
		=
		\left[
		\begin{array}{cccc}
		c_{11} & c_{12} & \cdots c_{1k}\\
		c_{21} & c_{22} & \cdots c_{2k}\\
		\vdots \\
		c_{m1} & c_{m2} & \cdots c_{mk}\\
				
		\end{array}		
		\right]						
		\]
		其中元素$c_{ij}$的值为
		$$
		c_{ij} = \sum_{r=1}^{n}a_{ir}b_{rk}
		$$
		如果我们把一个矩阵的每一行看做一个有限维向量(称为行向量),也把每一列看做一个
		有限维向量(称为列向量)。再考虑向量的数量积的定义,如果
		$$
		\vec{a} = (a_1,a_2,\cdots,a_n)
		$$
		$$
		\vec{b} = (b_1,b_2,\cdots,b_n)
		$$
		那么向量$\vec{a}$与$\vec{b}$的数量积为:
		$$
		\vec{a}\cdot\vec{b} = a_1b_1+a_2b_2+\cdots+a_nb_n=\sum_{i=1}^{n}a_ib_i
		$$
		现在我们可以给予矩阵乘法的一种比较直观的解释,积矩阵的元素$c_{ij}$是来自第一个因子
		矩阵的第$i$个行向量和第二个因子矩阵的第$j$个列向量的数量积;现在我们明白了为什么要求
		第一个矩阵的列数要与第二个矩阵的行数相等了,因为只有这样,才能保证矩阵$A$的行向量的维
		数与矩阵$B$的列向量的维数相同,而只有维数相同的向量才能进行数量积。
		\end{enumerate}
		
	\subsection{矩阵的现实及其运算意义}
		\subparagraph{矩阵的代数意义}
		前面已经提到,矩阵是从线性方程组中抽象出来的数学对象,它最重要的运用就是简化线性代数方
		程组的求解。对于每一个矩阵$A \in M_{m \times n}$,$A$代表某个线性方程组的系数矩
		阵,看看前述“矩阵的来源”一段中引入的“方块”,左侧称为“系数矩阵”,如果把右侧的
		$n\times 1$的矩阵(实际上它是一个$m$维列向量)“粘贴到”$A$的最右侧,即
		$$
		\left[
		\begin{array}{ccccc}
		a_{11} & a_{12} & \cdots & a_{1n} & b_1 \\
		a_{21} & a_{22} & \cdots & a_{2n} & b_2 \\
		\vdots & \vdots & \cdots & \vdots & \vdots\\
		a_{m1} & a_{m2} & \cdots & a_{mn} & b_m
		\end{array}
		\right]
		$$
		我们称新得到的矩阵为系数矩阵$A$的“增广矩阵”,每一个增广矩阵完全表征了一个线性方程
		组。这种形式很方便于利用高斯-若尔当消元法求解。\\
		\indent
		高斯-若尔当(Gauss-Jordan)消元法是判定线性方程组解集的性态,求解线性方
		程组,求矩阵的秩,求方阵的行列式、逆矩阵的极有效的方法。后面我们会详细解释它,此时
		我们只是先了解它的存在。\\
		\indent
		进一步,在这种代数意义下我们来研究矩阵的乘法的比较明显的意义的例子:我们考虑系数矩阵
		$A$,
		同时定义一个$n\times 1$的矩阵(一个$n$维列向量)
		$$		
		X=
		\left[
		\begin{array}{rl}
		x_1\\
		x_2\\
		\vdots \\
		x_n
		\end{array}
		\right]
		$$
		称为未知量矩阵($n$维向量),我们研究$A\cdot X$,显然我们可以得到一个$m$维的列向
		量,其每一个分量都是一个方程式的等号的左边部分,如果我们将每一个方程式等号的右边部分
		也写成一个$m$维列向量$B$
		$$		
		B=
		\left[
		\begin{array}{rl}
		b_1\\
		b_2\\
		\vdots \\
		b_m
		\end{array}
		\right]
		$$
		那么
		$$
		A \cdot X = B
		$$
		就完全是代数方程组的简写。
		
		\subparagraph{矩阵的几何意义}
		在线性代数中,一个矩阵总是会和线性空间、线性变换、空间的基等概念联系起来。此阶段我仅仅
		介绍而不详细解释,它涉及的基本概念太多。\\
		\indent
		我们对笛卡尔坐标很熟悉,对其中的向量的概念也不陌生;我们知道,在一个$n$维欧几里得空间
		(可以理解为有$n$个坐标轴的笛卡尔坐标系),每一个向量都可以表示成一个$n$维数组,这样
		如果我们记和每个坐标轴$x_i$平行的单位向量(长度为1的向量)为$\vec{e_i}$,那么
		$\vec{e_i}$可以表示成这样的数组:
		$$
		\vec{e_i} = (\underbrace{0,0,\cdots,0}_{i-1},1,0,\cdots,0)
		$$
		回想向量的加法以及向量与实数的乘法,我们就可以通过$\vec{e_1},\vec{e_2},\cdots
		,\vec{e_n}$的线性组合表示一切$n$维空间中的向量了。我们知道任意一个$n$维空间中的向量
		可以写成$\vec{v}=(v_1,v_2,\cdots,v_n)$,于是:
		\begin{align*}
		\vec{v} &= (v_1,v_2,\cdots,v_n)\\
				&= (v_1,0,0,\cdots,0)+(0,v_2,\cdots,0)+\cdots+(0,\cdots,0,v_n)\\
				&=v_1(1,0,\cdots,0)+v_2(0,1,0,\cdots,0)+\cdots+v_n(0,\cdots,0,1)\\
				&=v_1\vec{e_1}+v_2\vec{e_2}+\cdots+v_n\vec{e_n}
		\end{align*}
		于是,任何一个向量都可以通过这$n$个向量的线性组合来表示。我们把这样的$n$个向量的集合
		称为$n$维空间的一组基。\\
		\indent
		当然空间的基的形式不止一种,我们可以把其中一个$\vec{e_i}$换成$\vec{e_i}'=-\vec{e_i}=(0,0,\cdots,-1,0,\cdots,0)$,甚至换成另外它和另外任意一个向量$\vec{v}$的和,
		我们依然可以保证空间中的任意一个向量都能通过他们组合出来。所以,空间的基的形式并不唯一,但是可以确定的是,基的个数一定是$n$,否则一定存在不能用他们表示的向量,比如$\vec{e_n}$就不能用前面的$\vec{e_1},\vec{e_2},\cdots,\vec{e_{n-1}}$表示出来,因为这些向量的最后一个坐标值总是0,而$\vec{e_n}$的最后一个坐标值是1。\\
		\indent
		值得强调的是,并不是任意$n$个向量都能称为一组基,我们考虑如下$n$个向量,$\vec{e_1},2\vec{e_1},\cdots,n\vec{e_1}$,这样的$n$个向量的组合,用$\vec{v_i}$表示$i\vec{e_1}$
		$$
		\sum_{i=1}^{n}a_i\vec{v_i}=(\sum_{i=1}^{n}ia_i)\vec{e_1}
		$$
		这样的向量总是在坐标轴上,除此之外的点不可能由它们的组合表示出来。\\
		\indent
		而矩阵就相当于是这样的向量组的集合,方阵(行列数相同的矩阵)则有可能成为一个空间的基,
		矩阵的乘法就相当于对这些向量的线性组合,而成为一些新的向量集合。对于方阵,如果它包含的向量能够成为一个空间的基,那么他也可以作为一个坐标变换的方法存在(因为从一个坐标系到另一个坐标系的变换,总是对应于$n$次空间基的线性组合而成为$n$个新的基)。这里要再多指出一点,空间坐标变换指的是同一个向量在该空间的不同的基下的表示形式,向量本身并不改变,变换的只是表示形式;比如,对于同一个图形,如果我们把坐标轴的每个刻度的值增加到原来的2倍,但是图形本身在纸上并不改变,那么,得到的在新的意义下的坐标表示的每个分量都将会是原来的2倍;于是我们得到了同图形在新坐标下的表示方式。详细的过程我们会在今后进行。\\
		\subparagraph{小结}
		因此,一个数学对象的实际意义会根据所处情况的不同采取不同的解释,这种行为称为模型化,对于纯粹的数学概念,他们本身是没有任何实际上的意义的,只有当我们对其进行某种解释后,他们的意义就能显现出来;采用不同的解释,数学理论里面的每一个纯概念就会得到不同的实际意义。关于我们如何提取到这些纯粹的数学概念的,又是如何能对他们加以解释的原因,涉及很深层次的逻辑学和哲学问题,讨论极其复杂,你有兴趣的话,我再给你深入解释。

\subsection{高斯-若尔当消元法(Gauss-Jordan)}
	\subparagraph{在求解方程组时的运用}
	对于一个具有$n$个变量的$x_1,x_2,\cdots,x_n$的线性方程组,对任意的变量$x_i$我们总能找到至少一个方程组中的方程式,其中$x_i$的系数不为0;否则,如果对于任意的方程式$x_i$的系数都为0,那么这个变量就是多余的,不应当被写出。我们先写出这个方程组和它的增广矩阵(请回想增广矩阵的定义)。
	$$
		\left\{
		\begin{array}{rl}
		&a_{11}x_1 + a_{12}x_2 + \cdots + a_{1n}x_n = b_1\\
		&a_{21}x_1 + a_{22}x_2 + \cdots + a_{2n}x_n = b_2\\
		&\vdots \\
		&a_{m1}x_1 + a_{m2}x_2 + \cdots + a_{mn}x_n =b_m
		\end{array}
		\right.
	$$
	$$
		\left[
		\begin{array}{ccccc}
		a_{11} & a_{12} & \cdots & a_{1n} & b_1 \\
		a_{21} & a_{22} & \cdots & a_{2n} & b_2 \\
		\vdots & \vdots & \cdots & \vdots & \vdots\\
		a_{m1} & a_{m2} & \cdots & a_{mn} & b_m
		\end{array}
		\right]
	$$
	在求解线性方程组时,我们知道,将两个等式相加不改变方程组的解;在任意一组等式的等号两边同时乘以一个不为0的常数,也不改变方程组的解;当然,交换两个等式的位置,也不会改变方程组的解。对于前两点,我们很容易就能延伸到对增广矩阵的操作:将某一行加到另一行,矩阵对应的方程组的解保持不变,称为在解的意义下,变换后的矩阵和变换前的矩阵等价;将某一行乘以一个非零常数也能得到等价矩阵,结合这两点,我们可以断言,将矩阵的某一行的任意倍加到另一行,将得到等价矩阵;以及交换两行的位置,同样可以得到等价矩阵。\\
	\indent
	这三种操作,对应了矩阵的基础操作,今后会时常用到。现在,我们详细述说如何实现矩阵的高斯消元法,我们前面已经提到,高斯-若尔当消元法不仅仅可以用求解线性方程组,还可以进行很多其他的应用。我们先详细地介绍如何在理论层面中进行这种方法,然后在以Python语言为原型,详细讲解如何实现一个矩阵类,以及实现这些运算和操作。\\
	\indent
	首先我们对第一个变量$x_1$进行考量,如前述的约定,我们总可以找到一个包含$x_1$非零系数的方程组,不是一般性,我们可以假定第一行已经是这样的方程式,如若不然,我们可以交换两行,让第一行的$x_1$系数总不为零,即我们已经得到断言$a_11 \neq 0$。接下来,我们可以进行对第2至$n$行进行这样的操作,即让他们加上第一行的某个倍数得到一个新方程式,从而使分别他们的$x_1$系数为0;即:
	\begin{align*}
	a_{i1}x_1+a_{i2}x_2 + \cdots + a_{1n}x_n &= b_i\\
	+\\
	-\frac{a_{i1}}{a_{11}}(a_{11}x_1+a_{12}x_2+\cdots+a_{1n}) &= -\frac{a_{i1}}{a_{11}}b_1
	\end{align*}
	我们可以得到新的等式:
	$$
	0{x_1}+a_{i2}^*x_2+\cdots+a_{in}^*x_n = b_i^*
	$$
	其中$a_{ij}^*=a_{ij}-\frac{a_{i1}}{a_{11}}a_{1j},b_i^*=b_i-\frac{a_{i1}}{a_{11}}b_1$
	以同样的方式进行增广矩阵的操作,我们可以得到等价矩阵:
	$$
		\left[
		\begin{array}{ccccc}
		a_{11} 	& a_{12} & \cdots & a_{1n} & b_1 \\
		0 		& a_{22}^* & \cdots & a_{2n}^* & b_2^* \\
		\vdots 	& \vdots & \cdots & \vdots & \vdots\\
		0 		& a_{m2}^* & \cdots & a_{mn}^* & b_m^*
		\end{array}
		\right]
	$$
	至此我们已经把第一列的第一行之下全部变为0,同时保证了矩阵的等价性;现在,我们保持第一行不动,对第二行以及之后的所有行执行同样的操作,意即,首先判断第二行的第二个系数是否为0,如果是,那么判断在剩下的行中是否存在第二个系数不为零的行,如果有,那么就交换他们的行,然后进行上述操作;此时我们可以得到等价矩阵的形式如下:
	
	如果所有的$a_{22}^*,a_{32}^*,\cdots,a_{m2}^*$都是零,那么我们就得到行
\end{document}