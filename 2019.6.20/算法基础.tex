\documentclass[a4paper,12pt]{ctexart}
\usepackage{amsmath,amssymb,amsthm}
\usepackage{indentfirst}
\setlength{\parindent}{2em}
\usepackage{listings}
\usepackage{color}

\definecolor{dkgreen}{rgb}{0,0.6,0}
\definecolor{gray}{rgb}{0.5,0.5,0.5}
\definecolor{mauve}{rgb}{0.58,0,0.82}

\lstset{ %
  language=Octave,                % the language of the code
  basicstyle=\footnotesize,           % the size of the fonts that are used for the code
  numbers=left,                   % where to put the line-numbers
  numberstyle=\tiny\color{gray},  % the style that is used for the line-numbers
  stepnumber=2,                   % the step between two line-numbers. If it's 1, each line 
                                  % will be numbered
  numbersep=5pt,                  % how far the line-numbers are from the code
  backgroundcolor=\color{white},      % choose the background color. You must add \usepackage{color}
  showspaces=false,               % show spaces adding particular underscores
  showstringspaces=false,         % underline spaces within strings
  showtabs=false,                 % show tabs within strings adding particular underscores
  frame=single,                   % adds a frame around the code
  rulecolor=\color{black},        % if not set, the frame-color may be changed on line-breaks within not-black text (e.g. commens (green here))
  tabsize=2,                      % sets default tabsize to 2 spaces
  captionpos=b,                   % sets the caption-position to bottom
  breaklines=true,                % sets automatic line breaking
  breakatwhitespace=false,        % sets if automatic breaks should only happen at whitespace
  title=\lstname,                   % show the filename of files included with \lstinputlisting;
                                  % also try caption instead of title
  keywordstyle=\color{blue},          % keyword style
  commentstyle=\color{dkgreen},       % comment style
  stringstyle=\color{mauve},         % string literal style
  escapeinside={\%*}{*)},            % if you want to add LaTeX within your code
  morekeywords={*,...}               % if you want to add more keywords to the set
}
\begin{document}
	\subsection{记号定义}
		\textbf{定义1:}如果存在正的常数$c$和$n_0$使得,对于任何$N\geqslant n_0$时总有$T(N)
		\leqslant c f(N)$,那么就使用记号$T(N)=O(f(N))$。\\
		\indent
		\textbf{定义2:}如果存在正的常数$c$和$n_0$使得,对于任何$N\geqslant n_0$时总有$T(N)
		\geqslant cg(N)$,那么就使用记号$T(N)=\Omega(g(N))$。\\
		\indent
		\textbf{定义3:}如果我们有关系式$T(N)=O(h(N))$,并且同时还具有关系式$T(N)=
		\Omega(h(N))$,那么我们使用记号$T(N)=\Theta(h(N))$。\\
		\indent
		\textbf{定义4:}如果我们具有关系$T(N)=O(p(N))$,但是不具有关系$T(N)=\Theta(p(N))$,
		那么我们使用记号$T(N)=o(p(N))$。\\
		\indent
		此处的记号定义非常类似于数字的比较,我们可以很简单地将定义1类比于数字比较的小于或等于,意即
		如果我们有关系$T(N)=O(f(N))$,那么我们可以“认为”$T(N)\leqslant f(N)$,同样地,对于定
		义2,我们类比于关系小于或等于;显然定义三就类比于等于,而定义4可以类比于小于。注意,此处的
		出现的$T(N),f(N),g(N),p(N)$等,表明是一些关于$N$的函数,而$N$的取值一般限于整数。在算法
		分析中,我们会使用这些记号来度量算法之间的增长关系,即比较算法的运算时间的复杂程度。\\
		\indent
		这里的定义,直观地看来,与序列极限的定义十分类似;事实上,它们分别就是序列极限之比的另一种表
		述方式。我们来观察这样的极限表达式
		$$
		\lim_{n \rightarrow \infty}\frac{T(N)}{f(N)}=r
		$$
		如果$r\\=0$,那么我们可以断言$T(N)=o(f(N))$,如果$r=c(0<c<\infty)$,那么我们可以得到
		$T(N)=\Theta(f(N))$,如果$r=0$,我们有$f(N)=o(T(N))$。
	\subsection{T(N)函数的意义}
		\indent 在算法分析中,我们不可能精确地估计算法需要执行的时间,这是因为,首先我们无法预计该
		算法会以什么样的编程语言形式,在什么样的计算机上运行;即便是我们在同一台计算机上进行编译,编
		译器的发行商甚至版本的不同,也会造成效率不一的情况。所以,精确预计算法的运行时间是不可能的,
		但是,一个算法可能的运行时间快慢确实是可以量化估计的,但不是以确定时间的形式,我们会代之以其
		他类似的形式——基本运算次数——来估计算法的运行效率。$T(N)$函数,就是这样一种函数,对于一个输
		入规摸$N$,我们将会得到怎样量级的基本运算次数。\\
		\indent
		举例说明,如果$T(N)=\Theta(N^3)$,此时,如果我们的输入规摸是300,那么算法就需要$N^3$量
		级的运算次数,但是算法可能不一定就恰好进行$300^3$次基本运算,可能是它的某个常数倍,或者还有
		更多的出入,但是,当输入规摸$N$增加到非常大时,算法所需的基本运算次数会越来越趋近于$N^3$。
		\\
		\indent
		我们进行算法分析时,一般讨论一个算法对于输入规模$N$的最坏情况,有时候我们也会研究算法的平均
		运算时间,不过这不总是可行的,因为对于很多算法,我们无法估计平均情况,甚至有时候我们根本无法
		定义所谓的平均运算时间。这样虽然我们可能无法准确估计运行所需时间,但是我们可以给出一个上界,
		意味着对于这样的输入规摸,最坏的情况下,所需的最大运行时间不会超过我们得到的结果,但是可以
		小于这个结果;意味着我们可以估计,在这样的情况下,算法最久需要多长时间运行,这样也是很有用的
		情形。
	\subsection{算法分析例子}
		\indent \textbf{例子1:} 我们现在来对比两个排序算法的实例。
		\\ \indent \textit{插入排序}:分析算法首先给出代码
		\begin{lstlisting}[title=插入排序, frame=shadowbox]
		#include <iostream>
		#include <random>
		#include <time.h>
		using namespace std;

		int main()
		{
    			int n = 0;
    			cout << "Please enter number of random numbers:";
    			cin >> n;
    			int *p = new int[n];
    			srand(time(NULL));
    			for(int i=0; i < n; i++){
        		p[i] = rand() % (2*n);
    		}

    			cout << "\n";
    			for(int i = 0; i < n; i++){
        		cout << p[i] << " ";
    		}
    			cout << "\n";

    			for(int i = 1; i < n; i++){
        		int key = p[i];
        		int j = i - 1;
        		while(j > -1 && p[j] > key){
            		p[j+1] = p[j];
            		j -= 1;
        		}
        		p[j+1] = key;
    		}

    		cout << "\n";
    		for(int i = 0; i < n; i++){
        		cout << p[i] << " ";
    		}
    		cout << "\n";
		}
		
		\end{lstlisting}

		\indent
		我们假定每个基本命令都将消耗一个单位的时间,如果遇到函数调用,除去调用的时间开销
		一个单位外,函数内部的时间开销也要进行计算。依照此约定,我们来分析这个代码,对于输入规摸是
		$N$时,需要消耗的时间。当然,在实际的计算机中,不可能每个基本命令只消耗一个单位的时间,而且
		在实际的计算机中,一个单位的时间没有意义,我们只是为了分析算法的时间复杂性而引入的这样一种
		度量方式,它会和实际运算时间成正相关关系。\\
		\indent
		首先我们看看第8~12行代码,无论对于多大的输入规摸$N$,我们都将执行5次基本命令,意味着我们会
		消耗(在理论算法上)5个单位时间;
		$$
		T_1 = 5
		$$
		而第13——15行代码是一个for循环,此时我们预计最坏的情形是要运行$N$次,而初始化一个内部循环变
		量i需要消耗一个时间单位,每次运行时将i与n进行比较需要消耗一个时间单位,共最多$N$次,而i在每
		次内for循环部语句运行完成后需要自增1,每次也需要消耗1个单位时间,共最多$N$次,因此;
		$$
		T_{2_1}=2N+1
		$$
		再向内分析,这个for循环包含一个赋值语句,显然它们消耗时间多于一个单位,因为它同时包含多个操作
		这些操作有,取随机数rand(),乘法2*n,以及求余数\%,数组下标操作,以及赋值;这些操作每次需要
		5个单位时间,因此经过$N$次循环需要的总时间为
		$$
		T_{2_2}=5N
		$$
		从而对于整个循环我们有
		$$
		T_2 = T_{2_1}+T_{2_2} = 2N+1+5N=7N+1
		$$
		继续我们的分析,接下来第17行是一个流输出操作,我们认为它需要一个单位时间,因此
		$$
		T_3 = 1
		$$
		此处又是一个for循环,从而基本时间量同$T_{2_1}$,而内部语句不同;此时的内部语句实质上是两次
		流输出操作,因此它每次需要2个单位时间,从而
		$$
		T_4 = 2N+1+2N = 4N+1
		$$
		类似的,对于第21行,有
		$$
		T_5 = 1
		$$
		第23——31行,是算法的核心部分,我们对它进行分析;首先是一个for循环,它需要基本时间量(最坏的
		情况)是2N(因为是从i=1开始,比前述情况少执行一次,故而少1),然后内部代码分为三部分,前两部
		分是定义变量,但是操作并非单一的,24行有一个赋值操作与数组下标操作,消耗2个时间,第25行有一个
		赋值操作与减法操作,也需要消耗2个时间,故而这一部分将消耗
		$$
		t_{6_1} = (N-1)(2+2) = 4N-4
		$$
		接下来是一个while循环,在最坏的情况下,循环每次判断需要两次比较操作以及一次求逻辑与的操作,因
		此在最坏的情况下,每次循环判断需要3个单位时间,而在最坏的情况下,显然需要进行i-(-1)=i+1次判
		断,至少在最后一次判断时条件必然为假,否则将造成死循环,而我们分析代码,它将不会死循环,因为每
		次循环进行,j至少会少一个,而根据自然数的子序列必然存在有限下界的公理,因而循环次数必然为有限
		次,所以while循环体内部至多会执行i次。再看看while循环的内部情况,简要分析,我们知道第27行需
		要进行两次下标操作与一次赋值操作,28行需要一次自减操作,每次循环需要执行3次基本操作,因而需要
		3个单位时间,从而整个while循环(在每次for循环内)需要至多3i个单位时间,从而对于全部的for循
		环而言,第二部分,即while循环部分,共需要时间
		$$
		T_{6_2} = \sum_{i=1}^{N-1}3i=3\sum_{k=1}^{N-1}i=3\frac{(N-1)(N-1+1)}{2}=
		\frac{3}{2}N(N-1) 
		$$
		第三部分只有一个语句,但是每次它都需要两个单位时间,因为需要下标操作与赋值操作,因此
		$$
		T_{6_3}=2(N-1)=2N-2
		$$
		因此,这整个for循环需要最多
		$$
		T_6 = T_{6_1}+T_{6_2}+T_{6_3}=4N-4+\frac{3}{2}N(N-1)+2N-2=\frac{3}{2}N^2+
		\frac{9}{2}N-6
		$$
		对于代码余下的部分,我们可以很轻松地发现
		$$
		T_7=1+2N+1+3N=5N+1
		$$
		对于整个算法的运行时间上界,显然有
		\begin{align*}
				T(N)&=\sum_{i+1}^{7}T_i=5+7N+1+1+4N+1+1+4N-4+\frac{3}{2}N^2+\frac{9}
				{2}N-6+5N+1\\
				&=\frac{3}{2}N^2+5N-\frac{9}{2}N\\
				&=\frac{3}{2}N^2+\frac{1}{2}N\\
				&=\frac{3N^2+N}{2}
		\end{align*}
		这样,根据前述定义我们可以很显然地看出来,有关系
		$$
		T(N) = O(N^2)
		$$
		因此我们已经得到了这个算法的复杂性,称为“$O(N^2)$”的。
		
		\subsection{算法分析法则}
		\indent
		我们在定义算法时间复杂性时(那些$T(N)=O(f(N)),\Omega(g(N)),\Theta(h(N))$),已经分析了
		它们和极限的内在关系,这是当$N$足够大时(大到能满足我们形容的性质时,因为这些表达式具有这样
		的性质,即存在某些极限,所以必然能在某个值,从那样的$N$之后的一切N都满足某种性质),复杂性函
		数的值的最主要部分,忽略其余的与它相比微不足道的部分(即除以它之后,分式表达式的极限将为0),
		并且,不看主要部分的系数,只看它的量级,我们就会得到算法在足够大的输入规模下的时间复杂性的渐进
		方式。\\
		\indent
		在这种意义下,我们对每个算法都进行例子中的分析将是及其不明智的,因为,有很多操作,它们是固定大
		小,这些值在充分大的输入规模下是微不足道,比如上述的$T_3$与$T_5$,以及很多情况下,系数也是不
		重要的,上述中一切$T_i$的系数,这些都没必要进行分析。进而我们发现,在上述算法中,存在一个for
		循环和while循环的嵌套,这里的复杂性估算得到将是$N^2$量级的,因为每次循环在最坏的情况下,必然
		各自不会超过$N$次,因此对比与其他部分,在充分大的$N$的情况下,只有这里最重要,从而得到
		$T(N)=O(N^2)$。\\
		\indent
		现在我们来简述一些一般的法则:
		\\
		\indent \textit{法则1	——for循环}:
		\\
		\indent 
\end{document}