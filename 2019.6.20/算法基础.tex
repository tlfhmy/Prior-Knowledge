\documentclass[a4paper,12pt]{ctexart}
\usepackage{amsmath,amssymb,amsthm}
\usepackage{indentfirst}
\setlength{\parindent}{2em}
\begin{document}
	\subsection{记号定义}
		\textbf{定义1:}如果存在正的常数$c$和$n_0$使得,对于任何$N\geqslant n_0$时总有$T(N)
		\leqslant c f(N)$,那么就使用记号$T(N)=O(f(N))$。\\
		\indent
		\textbf{定义2:}如果存在正的常数$c$和$n_0$使得,对于任何$N\geqslant n_0$时总有$T(N)
		\geqslant cg(N)$,那么就使用记号$T(N)=\Omega(g(N))$。\\
		\indent
		\textbf{定义3:}如果我们有关系式$T(N)=O(h(N))$,并且同时还具有关系式$T(N)=
		\Omega(h(N))$,那么我们使用记号$T(N)=\Theta(h(N))$。\\
		\indent
		\textbf{定义4:}如果我们具有关系$T(N)=O(p(N))$,但是不具有关系$T(N)=\Theta(p(N))$,
		那么我们使用记号$T(N)=o(p(N))$。\\
		\indent
		此处的记号定义非常类似于数字的比较,我们可以很简单地将定义1类比于数字比较的小于或等于,意即
		如果我们有关系$T(N)=O(f(N))$,那么我们可以“认为”$T(N)\leqslant f(N)$,同样地,对于定
		义2,我们类比于关系小于或等于;显然定义三就类比于等于,而定义4可以类比于小于。注意,此处的
		出现的$T(N),f(N),g(N),p(N)$等,表明是一些关于$N$的函数,而$N$的取值一般限于整数。在算法
		分析中,我们会使用这些记号来度量算法之间的增长关系,即比较算法的运算时间的复杂程度。\\
		\indent
		这里的定义,直观地看来,与序列极限的定义十分类似;事实上,它们分别就是序列极限之比的另一种表
		述方式。我们来观察这样的极限表达式
		$$
		\lim_{n \rightarrow \infty}\frac{T(N)}{f(N)}=r
		$$
		如果$r=0$,那么我们可以断言$T(N)=o(f(N))$,如果$r=c(0<c<\infty)$,那么我们可以得到
		$T(N)=\Theta(f(N))$,如果$r=0$,我们有$f(N)=o(T(N))$。
	\subsection{T(N)函数的意义}
		\indent 在算法分析中,我们不可能精确地估计算法需要执行的时间,这是因为,首先我们无法预计该
		算法会以什么样的编程语言形式,在什么样的计算机上运行;即便是我们在同一台计算机上进行编译,编
		译器的发行商甚至版本的不同,也会造成效率不一的情况。所以,精确预计算法的运行时间是不可能的,
		但是,一个算法可能的运行时间快慢确实是可以量化估计的,但不是以确定时间的形式,我们会代之以其
		他类似的形式——基本运算次数——来估计算法的运行效率。$T(N)$函数,就是这样一种函数,对于一个输
		入规摸$N$,我们将会得到怎样量级的基本运算次数。\\
		\indent
		举例说明,如果$T(N)=\Theta(N^3)$,此时,如果我们的输入规摸是300,那么算法就需要$N^3$量
		级的运算次数,但是算法可能不一定就恰好进行$300^3$次基本运算,可能是它的某个常数倍,或者还有
		更多的出入,但是,当输入规摸$N$增加到非常大时,算法所需的基本运算次数会越来越趋近于$N^3$。
\end{document}